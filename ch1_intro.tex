\chapter{Introduction}
\section{A short history of smart meters}

The electrical grid and electrification history goes back to more than hundred years ago, when the electricity was usually produced near where it was going to be consumed. But later electricity utility companies found out that with ever growing request for electricity it would be more economical and beneficial to move to a more centralized solution. Then with the help of electrical grid it was possible to produce electrical energy in different places, combine them together to balance load and then transfer it to another place further away from the production plant to deliver it to the customer and consumer. The electrical grid technology was improved over time but its basic was the same.\par
But by the beginning of the 21th century new and more environment friendly energy sources, like solar and wind energy, was becoming popular and at the same time more and more electrical devices were being used by people. And the same time little by little traditional IT systems were being used in industrial systems. These had an enormous consequence in the way the electric companies were working and thus making the old ways of controlling and distributing obsolete or very hard to do.\par 
This was where the smart grid was introduced, with the aim to cope with the new requirements imposed by the challenges facing the grid companies, and to modernize the way electricity was being produced and distributed. But still something was missing in this chain and it was the ever growing usage of customers and unavailability of tools to measure the consumption in details and react to it in case of emergency to balance the load or minimize the damage to the electrical system. As well as the ability for the customers to contribute to the production of energy by having installed solar panels that it was more affordable now.\par 
And the answer to these challenges was smart meters or \ac{ami}. Introduction of the smart meters revolutionized the way grid companies were interacting with the customers and their billing methods. Now with the two-way communication they were able to disconnect the energy whenever they wanted and also it was possible to measure the consumption every hour and charge the customers with the customers based on the price of electricity in that hour and motivate them not to use a lot of electricity during peak hours to ease the pressure on the system. On the other hand customers were also able to benefit from cheaper energy during certain hours and also take benefit from the smart home services which was enabled by having smart meters.

\section{Problem description}

While the advantages that smart meters bring with themselves seems to be promising, there are some concerns that are not yet completely resolved or answered. The biggest concern for the consumers is preserving their privacy, since the information gathered and transmitted by the meters may contain Personally identifiable information (\ac{pii}) which can be misused by others. As well as information which might indirectly lead to some security concerns for the consumers, like for example helping the attackers to when their houses is empty so that they can go to the property and steal valuable items.\par
The other major concern which is important both for the consumers and grid companies is the security of smart meters. In many countries smart meters are equipped with a function which enables the companies to shut down the electricity. If malicious actors get access to this feature then they might be able to cause a major blackout which would not be easy to mitigate.\par 
All these concerns can be mitigated and minimized by analyzing these meters thoroughly through third parties and then publish the results to the public so that the consumers can be sure that neither their privacy nor their security is in danger by using the smart meters. But unfortunately as far as we know this generally has not happened yet and security by obscurity is still the major business model ruling the industry. We hope that by means of our research we can shed some light into this area.

\section{Motivation}

As mentioned in the previous section, the biggest motivation for us is helping both the consumers and the electricity companies to get a better understanding of the security of smart meters and in case of any weaknesses try to fix it before it is too late.\par 
We are also motivated to do this analysis because the government of Norway has decided to role out the smart meters in the whole country by the end of 2019, which means that several million smart meters are going to be installed in every household during the next two years. This is big opportunity for the grid companies to upgrade and renew their systems, but at the same time could mean big challenges ahead if we don't think that much about the security right now.\par 
On the other hand Norway is not the only country in the world which is going to implement these smart meters, there are several other countries which has either already started using these meters (like Sweden and Italy) or will do it during the next few years. This means that the results of our research could be also interesting and useful for a big number of people all over the world who are going to have a smart meter installed in their house. 


