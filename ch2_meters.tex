\chapter{Laws, regulations and standards}
In this chapter we will take a short look at laws and regulations regarding the smart meters in some countries, as well as some standards to get a bird's eye view on different requirements that the smart meter makers must take into account when building the devices and will try to look at some of the differences between these regulations.

\section{Norway}
In Norway according to paragraph 4-2-G of “Regulations governing metering, settlement, billing of network services and electrical energy”  \citep{lovdata} passed by the Norwegian Ministry of Petroleum and Energy, the smart meters must “provide security against misuse of data and unauthorized access to control functions”. But the burden of defining technical and detailed requirements to satisfy these requirements are put on the shoulders of The Norwegian Water Resources and Energy Directorate (\ac{nve}).\par
In September 2012 \ac{nve} published a “Guide to security in advanced metering and control systems” \citep{nve} which includes some guidelines that energy companies may follow in order to make sure that their smart meters are implemented in a secure manner, but again it is up to grid companies and smart meter vendors to sit together and decide to implement any of these suggestions according to their needs.

\section{Netherlands}
In the Netherlands, grid operators are responsible for installing and reading out smart meters and for offering services associated with market facilitation. The advanced metering infrastructure designed for this purpose has been set down in Netherlands Technical Agreement (NTA) 8130 \citep{dutch2} of the Netherlands Standardization Institute (NEN).\par
Netbeheer Nederland (the association in the energy sector representing the interests of network operators) decided to draw up a joint policy which all grid operators must comply with. The national ‘Smart Meter Privacy and Security Working Group’ has been set up for this purpose and is charged with developing requirements for managing the privacy and security risks relating to smart meters.\par
The Privacy and Security Working Group has defined a framework that serves as the foundation for securing the advanced metering infrastructure \citep{dutch1}. This foundation must safeguard the availability, integrity and confidentiality of information arising and minimize any damage caused by security incidents within the advanced metering infrastructure.

\section{United Kingdom}
The government wants energy suppliers to install smart meters in every home in England, Wales and Scotland. There are more than 26 million homes for the energy suppliers to get to, with the goal of every home having a smart meter by 2020, but smart meters aren’t compulsory and people can choose not to have one.\par
UK offers a certification scheme based on security characteristics defined by coordination of the Department of Energy and Climate Change, with cross-industry input, in support of the Smart Metering Implementation Program. The CPA Security Characteristics document \citep{uk1} describes requirements for assured Electricity Smart Metering Equipment (ESME) products for evaluation and certification under CESG’s Commercial Product Assurance (CPA) scheme \citep{uk2}.

\section{United States}
Support for the smart grid in the United States became federal policy with passage of the Energy Independence and Security Act of 2007 \citep{us1}. The law set out \$100 million in funding per fiscal year from 2008–2012, established a matching program to states, utilities and consumers to build smart grid capabilities, and created a Grid Modernization Commission to assess the benefits of demand response and to recommend needed protocol standards. The law also directed the National Institute of Standards and Technology to develop smart grid standards, which the Federal Energy Regulatory Commission (FERC) would then promulgate through official rule makings.\par
Smart grids received further support with the passage of the American Recovery and Reinvestment Act of 2009 \citep{us2}, which set aside \$4.5 billion of funding for Smart Grid development, deployment, and worker training. In 2014, U.S. electric utilities had about 58.5 million advanced smart metering infrastructure (AMI) installations. About 88\% were residential customer installations.\par
Cybersecurity Working Group (CSWG) of the Smart Grid Interoperability Panel (SGIP), a public-private partnership launched by NIST, in January 2010 published the first version of Guidelines for Smart Grid Cybersecurity. The three volumes that make up the initial set of guidelines are intended primarily for individuals and organizations responsible for addressing cybersecurity for Smart Grid systems and the constituent subsystems of hardware and software components. The last version is published in September 2014 \citep{us3}.