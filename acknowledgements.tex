\begin{acknowledgements}
Typically the structure moves from thanking the most formal support to the least formal thanks as detailed above–funders, supervisors, other academics, colleagues, and finally family. This makes sense according to the logic of incremental progression because the informal thanks to family are often the most heartfelt. Close family members are often the people who gave the most (although some supervisors are likely to feel this is not true).

It is important that a student acknowledges the formal carefully, though: any person or institution that has contributed funding to the project, other researchers who have been involved in the research, institutions that have aided the research in some way. They should also acknowledge proofreaders and editors. Such formal thanks are usually in the first paragraph or two.
\end{acknowledgements}